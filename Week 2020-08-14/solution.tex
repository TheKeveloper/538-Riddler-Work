\documentclass{article}
\usepackage[utf8]{inputenc}
\usepackage{hyperref}
\usepackage{ksty}

\newcommand{\Unif}{\text{Unif}}
\newcommand{\Bin}{\text{Binomial}}
\title{FiveThirtyEight Riddler 08-14-2020}
\author{Kevin Bi}
\date{August 2020}

\begin{document}

\maketitle
The below is a response to the FiveThirtyEight Riddler Classic for the \href{https://fivethirtyeight.com/features/are-you-hip-enough-to-be-square/}{week of 08-08-2020}:

We can first state this problem in a more clear way: given an interval $[0, 1]$ with three points distributed uniform $[0, 1]$, what is the expected length of the segment containing $0.5$?

First, define $L$ as the random variable for the length of the line containing 0.5. We are trying to find $E[L]$. Notice that $L$ is simply the difference between the greatest point less than 0.5 (or 0 if there are none) and the least point greater than 0.5 (or 1 if there are none). So, $E[L]$ is given by the expected value of the minimum point on the greater (right) side minus the maximum point on the lesser (left) side.
\image{plot.jpeg}

To condition on this fact, let $X$ be an r.v. of the number of points less than 0.5 out of the three. Then, we can use the law of total expectation to obtain,
\begin{align*}
    E[X] = \sum_{k = 0}^3 E[L | X = k]P(X = k)
\end{align*}
Now we can just solve for each of the terms on the RHS. First, notice that $X \sim \Bin(3, 1/2)$. This is because we can count each point less than 0.5 as a success, and each cut is uniformly distributed hence has a 0.5 chance of being on the left side. This gives us $P(X = k)$. 

Next, we want to solve for $E[X | L = k]$. First, notice that by symmetry, $E[X | L = 0] = E[X | L = 3]$, and $E[X | L = 1] = E[X | L = 2]$, because these are just flipping which side the points land on. We will find each of these values. Define $U(j, n)$ as the expectation of the $j$th order statistic on $n$ $\Unif(0, 0.5)$ points. By the formula for the expectation of the order statistic on a uniform to obtain,
\begin{align*}
    U(j, n) = \frac{(1/2) j}{n + 1}
\end{align*}

First, we find $E[X | L = 0]$. This means that all of the points are on the right of 0.5. So, the length is just the minimum point on the right. This is given by,
\begin{align*}
    E[X | L = 0] = \frac{1}{2} + U(1, 3) = \frac{1}{2} + \left(\frac{1}{2}\right) \frac{1}{4} = 5/8
\end{align*}

Next, we find $E[X | L = 1]$. This means that there is one point on the left, and two points on the right. We can just subtract the point on the left from the point on the right. The expected position of the point on the left is given by 0.25, which is the middle of $[0, 0.5]$. The expected position of the minimum point on the right is given by $0.5 + U(1, 2) = 0.5 + 0.5 (1/3)$. This yields,
\begin{align*}
    E[X | L = 1] = 0.5 + U(1, 2) - 0.25 = 5/12
\end{align*}

So, we have found all of the conditional expectations. Plugging these into the original sum along with the PMF of the binomial yields the solution as,
\begin{align*}
    E[X] = 15/32
\end{align*}

\end{document}
